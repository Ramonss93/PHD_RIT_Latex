\documentclass[11pt,reqno]{report}
\usepackage{geometry}                % See geometry.pdf to learn the layout options. There are lots.
%\usepackage[paper=a4paper,dvips,top=1.5cm,left=1.5cm,right=1.5cm,
%    foot=1cm,bottom=1.5cm]{geometry}


\geometry{letterpaper}                   % ... or a4paper or a5paper or ... 
%\geometry{landscape}                % Activate for for rotated page geometry
\usepackage[parfill]{parskip}    % Activate to begin paragraphs with an empty line rather than an indent
\usepackage{graphicx}
\usepackage{amssymb,amsmath}
\usepackage{mathtools}
\usepackage{epstopdf}
\usepackage{float}
\usepackage{color}
\usepackage{soul}
\usepackage{indentfirst}
\setlength{\parindent}{1cm}


\DeclareGraphicsRule{.tif}{png}{.png}{`convert #1 `dirname #1`/`basename #1 .tif`.png}

\let\stdsection\chapter  
\renewcommand\chapter{\newpage\stdsection}  




\title{Atmospheric Compensation for WorldView-2 Satellite and In-Water Component Retrieval}
\author{
	\textsc{J.A. Concha}%\thanks{Contact Author}
	\mbox{}\\
	Center for Imaging Science\\ 
	Rochester Institute of Technology\\
	Rochester, NY, \underline{USA}\\
	\mbox{}\\
	\normalsize
		\texttt{jxc4005@rit.edu}
	}
\date{}                                           % Activate to display a given date or no date
%\date{\today}

%%%%%%%%%%% for Appendix
\makeatletter
\newcommand\appendix@section[1]{%
  \refstepcounter{section}%
  \orig@section*{Appendix \@Alph\c@section: #1}%
  \addcontentsline{toc}{section}{Appendix \@Alph\c@section: #1}%
}
\let\orig@section\section
\g@addto@macro\appendix{\let\section\appendix@section}
\makeatother


%%%%%%%%%%%%%%

\usepackage{hyperref}
\hypersetup{
    bookmarks=true,         % show bookmarks bar?
    unicode=false,          % non-Latin characters in Acrobat�s bookmarks
    pdftoolbar=true,        % show Acrobat�s toolbar?
    pdfmenubar=true,        % show Acrobat�s menu?
    pdffitwindow=false,     % window fit to page when opened
    pdfstartview={FitH},    % fits the width of the page to the window
    pdftitle={WV-2 In Water Component Retrieval },    % title
    pdfauthor={Javier Concha},     % author
    pdfsubject={Subject},   % subject of the document
    pdfcreator={Creator},   % creator of the document
    pdfproducer={Producer}, % producer of the document
    pdfkeywords={keyword1} {key2} {key3}, % list of keywords
    pdfnewwindow=true,      % links in new window
    colorlinks=true,       % false: boxed links; true: colored links
    linkcolor=blue,          % color of internal links
    citecolor=green,        % color of links to bibliography
    filecolor=magenta,      % color of file links
    urlcolor=cyan           % color of external links
}

\usepackage[all]{hypcap} % to see figure with hyper ref

\setcounter{secnumdepth}{5}
\setcounter{tocdepth}{5}

\begin{document}


\maketitle

\pagenumbering{roman}

\chapter*{Abstract}
\addcontentsline{toc}{chapter}{Abstract}


In the present work, the WorldView-2 (WV2) capability for retrieving Case 2 water components is analyzed. The WV2 sensor characteristics, such as a 11-bit quantization, 8 bands in the VNIR and high Signal-to-Noise Ratio (SNR) make WV2 potentially suitable for a retrieval process. In the Case 2 water problem, the sensor-reaching signal due to water is very small when compared to the signal due to the atmospheric effects. Therefore, adequate atmospheric compensation becomes an important first step to accurately retrieving water parameters. The problem becomes more difficult when using multispectral imagery as there are typically only a handful of bands suitable for performing atmospheric compensation. In this work, we develop atmospheric compensation techniques designed specifically for the WV2 satellite, enabling it to be used for water constituent retrieval in both deep and shallow water. A look-up-table (LUT) methodology is implemented to retrieve the water parameters Chlorophyll, Suspended Materials, Colored Dissolved Organic Matter, bathymetry, bottom type and water clarity for a simulated case study. The in-water radiative transfer code HydroLight is used to simulate reflectance data in this study while the MODTRAN code is used to simulate atmospheric effects. The resulting modeled sensor-reaching radiance can be sampled to a WV2 sensor model to simulate WV2 image data. This data is used to test the described methodology. Finally, a sensitivity analysis is performed to evaluate how sensitive the retrieval process is to adequate atmospheric compensation.

\chapter*{Acknowledgements}

\tableofcontents


\listoffigures
\addcontentsline{toc}{chapter}{List of Figures}

\listoftables
\addcontentsline{toc}{chapter}{List of Tables}
%  \begin{equation}
%  {\bf X}[k+1]=A{\bf X}[k]+B{\bf u}[k]
%  %\label{stateSpaceForm1}
%  \end{equation}
%
%
%  \begin{figure}[H]
%  \centerline{
%    \includegraphics[width=120mm]{LUT8.png}
%  }
%  \caption{Replace text here with your desired caption.}
%  \label{overView}
%  \end{figure}

%% CHAPTER
\chapter{Introduction}
\pagenumbering{arabic}



\chapter{Objectives}
\section{Problem Statement}

\section{Statement of Objectives}

\section{Description of Tasks}
\subsection{Primary Requirements}
\subsection{Future Objectives}

\section{Contribution to Field}

%% CHAPTER
\chapter{Background  and Theory}
\section{Water Quality Parameters Retrieval}
\cite{Jensen}
\cite{Mustard2001}
\subsection{Governing Equation}

\hl{The comparison is performed in the reflectance domain.}

\hl{CHL, SM and CDOM explanation}

From the total energy coming from the sun, only approximately 1390 $\left[\frac{W}{m^2}\right]$ reaches the Earth's atmosphere \cite{Schott}. This integrated value is known as the \emph{exoatmospheric irradiance}, or $E_S'$, and represents the total energy per unit area just outside the Earth's atmosphere due to the solar energy. Recall that \emph{irradiance} is the rate at which the radiant flux ($\Phi$) is delivered to a surface ($A$), defined as

\begin{equation} \label{eq:irradiance}
E = \frac{d\Phi}{dA}   \indent   \indent  \left[\frac{W}{m^2}\right]  
\end{equation} 

So, the $E_S'$ is calculated assuming that the flux $\Phi$ comes from a point source at the center of the sun such that it would produce an existence at the sun's surface, producing a flux at the mean Earth-sun distance of 1390 $\left[\frac{W}{m^2}\right]$ . For the present work, it is more convenient to express the irradiance spectrally, or other words as a function of wavelength, so we can describe the energy at desired wavelength, or spectral band.

The light source, usually the sun, interact with the target and then reach the sensor. This interaction will help us to extract about the target, in this case the water body. That is why in order to understand how the water quality parameter are retrieved, first it is necessary to introduce the concept of sensor reaching radiance. The sensor reaching radiance is defined as the accumulation of photons at the front of a sensor that one wishes to collect in an effort to obtain information about the target \cite{Gerace}. The total sensor-reaching radiance is the sum of the radiances due to the individual solar and thermal paths.  \cite{Schott} shows that in the VNIR region (approximately 0.3-2.5 [$\mu m$]), the solar energy is so many orders of the magnitude higher than the self-emitted energy, so the thermal paths are negligible for this study. Also, we consider that radiance from the background is negligible because the water bodies are typically several kilometers wide. Assuming that the target is approximately Lambertian (radiance is equal in all directions), the total sensor-reaching radiance, $L$, is defined as

\begin{equation} \label{eq:gov1}
L(\lambda) = \frac{E'_S(\lambda)cos(\sigma')r(\lambda)\tau_1(\lambda)\tau_2(\lambda)}{\pi} +
                        \frac{E_{ds}(\lambda)r(\lambda)\tau_2(\lambda)}{\pi} + L_{us}(\lambda)
\end{equation} 
where:
\begin{tabbing}
\indent \indent \indent  $L(\lambda)$ \hspace{1mm}\=:  \indent \= total sensor-reaching radiance\\
\indent \indent \indent  $E'_S(\lambda)$\>: \>exoatmospheric spectral irradiance\\
\indent \indent \indent $\sigma'$\>:\>solar-zenith angle\\
\indent \indent \indent $r(\lambda)$\>:\>spectral reflectance target\\
\indent \indent \indent $\tau_1(\lambda)$\>:\>Sun-target path transmission of atmosphere\\
\indent \indent \indent $\tau_2(\lambda)$\>:\>target-sensor path transmission of atmosphere\\
\indent \indent \indent $E_{ds}(\lambda)$\>:\>solar downwelling irradiance\\
\indent \indent \indent $L_{u}(\lambda)$\>:\>solar upwelling irradiance\\
\end{tabbing}

Equation \eqref{eq:gov1} is the solar term of the "big equation" described by \cite{Schott}.

\subsection{Constituent Retrieval}

When imaging water body, a set of new paths need to be taken in account to determine the sensor-reaching radiance.
 \begin{figure}[H]
	\centering
    	\includegraphics[width=100mm]{/Users/javier/Desktop/Javier/MASTER_RIT/2011_THESIS/LaTeX/Thesis/Images/WaterColumn.eps}
 	\caption{Contributions to sensor-reaching radiance from the water column \label{fig:WaterColumn}}
  \end{figure}


Each sensor-reaching radiance curve is associated with a specific combination of water components (CHL, SM and CDOM). HidroLight provides us the \hl{remote sensing reflectance}, $R_{RS}$, which describes how much of the total incident downwelling irradiance is ultimately returned from a water column in a given direction, defined as

\begin{equation} \label{eq:Rrs}
R_{RS}(\theta,\phi,\lambda,z=a) = \frac{L(\theta,\phi,\lambda,z=a)}{E_d(\lambda,z=a)}   \indent   \indent  \left[\frac{1}{sr}\right]  
\end{equation} 
where:
\begin{tabbing}
\indent \indent \indent  $\theta$ \hspace{1.5mm}\=:  \indent \= sensor-zenith angle\\
\indent \indent \indent  $\phi$\>: \>sensor-azimuth angle\\
\indent \indent \indent $L$\>:\>water-leaving radiance\\
\indent \indent \indent $E_d$\>:\>total downwelling irradiance\\
\indent \indent \indent $\lambda$\>:\>wavelength dependent\\
\indent \indent \indent $a$\>:\>height just above the water's surface\\
\end{tabbing}



\section{Shallow Water Retrieval}
\section{LUT Method}
\subsection{Population}
\subsection{Optimization Algorithm  - lsqnonlin}

\section{State of the Research}

As shown in \cite{Gerace}
and \cite{Mobley} and \cite{Lesser}

\section{Atmospheric Compensation}

%% CHAPTER
\chapter{Methodology}
\section{WV-2 Sensor}
\subsection{Sensor Response}
\subsection{Adding Noise}
\subsection{Quantization}

\section{Retrieval Process}
\subsection{The Look-Up Table}

\subsubsection{Perfect Atmosphere Conditions}

\subsubsection{Real Atmosphere Conditions}



\section{Over-Water Atmospheric Compensation}
\subsection{The Empirical Line Method (ELM)}
\subsection{LUT Method}

%% CHAPTER
\chapter{Results}
\section{Synthetic Data}
\subsection{Deep Water}

  \begin{figure}[H]
	\centering
    	\includegraphics[width=100mm]{/Users/javier/Desktop/Javier/MASTER_RIT/2011_THESIS/LUT/LUT_1/Images/LUT1_120re.eps}
 	\caption{Error  \label{fig:errorLUT1}}
  \end{figure}

  \begin{figure}[H]
	\centering
    	\includegraphics[width=100mm]{/Users/javier/Desktop/Javier/MASTER_RIT/2011_THESIS/LUT/LUT_1/Images/TS1_120re.eps}
 	\caption{Error  \label{fig:errorLUT1}}
  \end{figure}
  
    \begin{figure}[H]
	\centering
    	\includegraphics[width=100mm]{/Users/javier/Desktop/Javier/MASTER_RIT/2011_THESIS/LUT/LUT_1/Images/TS1TOA120.eps}
 	\caption{Error  \label{fig:errorLUT1}}
  \end{figure}
  
      \begin{figure}[H]
	\centering
    	\includegraphics[width=100mm]{/Users/javier/Desktop/Javier/MASTER_RIT/2011_THESIS/LUT/LUT_1/Images/TS1TOA8.eps}
 	\caption{Error  \label{fig:errorLUT1}}
  \end{figure}
  
        \begin{figure}[H]
	\centering
    	\includegraphics[width=100mm]{/Users/javier/Desktop/Javier/MASTER_RIT/2011_THESIS/LUT/LUT_1/Images/TS1TOA8NQ.eps}
 	\caption{Error  \label{fig:errorLUT1}}
  \end{figure}
  
          \begin{figure}[H]
	\centering
    	\includegraphics[width=100mm]{/Users/javier/Desktop/Javier/MASTER_RIT/2011_THESIS/LUT/LUT_1/Images/TS1ELM.eps}
 	\caption{Error  \label{fig:errorLUT1}}
  \end{figure}
  
          \begin{figure}[H]
	\centering
    	\includegraphics[width=100mm]{/Users/javier/Desktop/Javier/MASTER_RIT/2011_THESIS/LUT/LUT_1/Images/DarkBrightpxre.eps}
 	\caption{Error  \label{fig:errorLUT1}}
  \end{figure}
  
  After ELM...

  \begin{figure}[H]
	\centering
    	\includegraphics[width=100mm]{/Users/javier/Desktop/Javier/MASTER_RIT/2011_THESIS/LUT/LUT_1/Images/SMrealretrievedELM.eps}
 	\caption{Error  \label{fig:errorLUT1}}
  \end{figure}



  \begin{figure}[H]
	\centering
    	\includegraphics[width=100mm]{/Users/javier/Desktop/Javier/MASTER_RIT/2011_THESIS/LUT/LUT_1/Images/CDOMrealretrievedELM.eps}
 	\caption{Error  \label{fig:errorLUT1}}
  \end{figure}
  
  
  
  \begin{figure}[H]
	\centering
    	\includegraphics[width=100mm]{/Users/javier/Desktop/Javier/MASTER_RIT/2011_THESIS/LUT/LUT_1/Images/CHLrealretrievedELM.eps}
 	\caption{Error  \label{fig:errorLUT1}}
  \end{figure}

As you can see in Figure \ref{fig:errorLUT1}...

  \begin{figure}[H]
	\centering
    	\includegraphics[width=100mm]{/Users/javier/Desktop/Javier/MASTER_RIT/2011_THESIS/LUT/LUT_1/Images/errorCHLSMCDOM.eps}
 	\caption{Error  \label{fig:errorLUT1}}
  \end{figure}

\subsection{Shallow Water}

As you can see in Figure \ref{fig:errorLUT2}...

  \begin{figure}[H]
	\centering
    	\includegraphics[width=100mm]{/Users/javier/Desktop/Javier/MASTER_RIT/2011_THESIS/LUT/LUT_2/figures/error_CDOMDepthBRELM.eps}
 	\caption{Error  \label{fig:errorLUT2}}
  \end{figure}

\subsection{Sensitivity Analysis}
\subsection{Adding a new band to WV-2}

\section{Real Data}

%% CHAPTER
\chapter{Summary and Recommendations}
\section{Summary}
\section{Future Work}

%% CHAPTER
\appendix
\chapter*{Appendix}
\addcontentsline{toc}{chapter}{Appendix}
\section{WV-2 Sensor Characteristics}
\section{HydroLight Specification}
\section{MODTRAN Specification}

  
%\bibliographystyle{ieeetr}
%\bibliographystyle{unsrtnat}
\bibliographystyle{apalike}

\bibliography{/Users/javier/Desktop/Javier/PHD_RIT/Latex/javier_bib}


\end{document}  