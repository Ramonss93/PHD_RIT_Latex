\documentclass[11pt,reqno]{report}
\usepackage{geometry}                % See geometry.pdf to learn the layout options. There are lots.
%\usepackage[paper=a4paper,dvips,top=1.5cm,left=1.5cm,right=1.5cm,
%    foot=1cm,bottom=1.5cm]{geometry}


\geometry{letterpaper}                   % ... or a4paper or a5paper or ... 
%\geometry{landscape}                % Activate for for rotated page geometry
\usepackage[parfill]{parskip}    % Activate to begin paragraphs with an empty line rather than an indent
\usepackage{graphicx}
\usepackage{amssymb,amsmath}
\usepackage{mathtools}
\usepackage{epstopdf}
\usepackage{float}
\usepackage{color}
\usepackage{soul}
\usepackage{indentfirst}
\setlength{\parindent}{1cm}


\DeclareGraphicsRule{.tif}{png}{.png}{`convert #1 `dirname #1`/`basename #1 .tif`.png}

\let\stdsection\chapter  
\renewcommand\chapter{\newpage\stdsection}  




\title{Atmospheric Compensation for WorldView-2 Satellite and In-Water Component Retrieval}
\author{
	\textsc{J.A. Concha}%\thanks{Contact Author}
	\mbox{}\\
	Center for Imaging Science\\ 
	Rochester Institute of Technology\\
	Rochester, NY, \underline{USA}\\
	\mbox{}\\
	\normalsize
		\texttt{jxc4005@rit.edu}
	}
\date{}                                           % Activate to display a given date or no date
%\date{\today}

%%%%%%%%%%% for Appendix
\makeatletter
\newcommand\appendix@section[1]{%
  \refstepcounter{section}%
  \orig@section*{Appendix \@Alph\c@section: #1}%
  \addcontentsline{toc}{section}{Appendix \@Alph\c@section: #1}%
}
\let\orig@section\section
\g@addto@macro\appendix{\let\section\appendix@section}
\makeatother


%%%%%%%%%%%%%%

\usepackage{hyperref}
\hypersetup{
    bookmarks=true,         % show bookmarks bar?
    unicode=false,          % non-Latin characters in Acrobat�s bookmarks
    pdftoolbar=true,        % show Acrobat�s toolbar?
    pdfmenubar=true,        % show Acrobat�s menu?
    pdffitwindow=false,     % window fit to page when opened
    pdfstartview={FitH},    % fits the width of the page to the window
    pdftitle={WV-2 In Water Component Retrieval },    % title
    pdfauthor={Javier Concha},     % author
    pdfsubject={Subject},   % subject of the document
    pdfcreator={Creator},   % creator of the document
    pdfproducer={Producer}, % producer of the document
    pdfkeywords={keyword1} {key2} {key3}, % list of keywords
    pdfnewwindow=true,      % links in new window
    colorlinks=true,       % false: boxed links; true: colored links
    linkcolor=blue,          % color of internal links
    citecolor=green,        % color of links to bibliography
    filecolor=magenta,      % color of file links
    urlcolor=cyan           % color of external links
}

\usepackage[all]{hypcap} % to see figure with hyper ref

\setcounter{secnumdepth}{5}
\setcounter{tocdepth}{5}

\begin{document}


\maketitle

\pagenumbering{roman}

\chapter*{Abstract}
\addcontentsline{toc}{chapter}{Abstract}


In the present work, the WorldView-2 (WV2) capability for retrieving Case 2 water components is analyzed. The WV2 sensor characteristics, such as a 11-bit quantization, 8 bands in the VNIR and high Signal-to-Noise Ratio (SNR) make WV2 potentially suitable for a retrieval process. In the Case 2 water problem, the sensor-reaching signal due to water is very small when compared to the signal due to the atmospheric effects. Therefore, adequate atmospheric compensation becomes an important first step to accurately retrieving water parameters. The problem becomes more difficult when using multispectral imagery as there are typically only a handful of bands suitable for performing atmospheric compensation. In this work, we develop atmospheric compensation techniques designed specifically for the WV2 satellite, enabling it to be used for water constituent retrieval in both deep and shallow water. A look-up-table (LUT) methodology is implemented to retrieve the water parameters Chlorophyll, Suspended Materials, Colored Dissolved Organic Matter, bathymetry, bottom type and water clarity for a simulated case study. The in-water radiative transfer code HydroLight is used to simulate reflectance data in this study while the MODTRAN code is used to simulate atmospheric effects. The resulting modeled sensor-reaching radiance can be sampled to a WV2 sensor model to simulate WV2 image data. This data is used to test the described methodology. Finally, a sensitivity analysis is performed to evaluate how sensitive the retrieval process is to adequate atmospheric compensation.

\chapter*{Acknowledgements}

\tableofcontents


\listoffigures
\addcontentsline{toc}{chapter}{List of Figures}

\listoftables
\addcontentsline{toc}{chapter}{List of Tables}
%  \begin{equation}
%  {\bf X}[k+1]=A{\bf X}[k]+B{\bf u}[k]
%  %\label{stateSpaceForm1}
%  \end{equation}
%
%
%  \begin{figure}[H]
%  \centerline{
%    \includegraphics[width=120mm]{LUT8.png}
%  }
%  \caption{Replace text here with your desired caption.}
%  \label{overView}
%  \end{figure}

\input{Introduction.tex}
\input{Objectives.tex}
\input{Background_and_Theory.tex}
\input{Methodology.tex}
\input{Results.tex}
\input{Summary}
\input{Appendix.tex}

  
%\bibliographystyle{ieeetr}
%\bibliographystyle{unsrtnat}
\bibliographystyle{apalike}

\bibliography{/Users/javier/Desktop/Javier/PHD_RIT/Latex/javier_bib}


\end{document}  