% !TEX root=Thesis_PhD.tex 
% the previous is to reference main.bib
%% CHAPTER
\begin{appendices}
% @@@@@@@@@@@@@@@@@@@@@@@@@@@@@@@@@@@@@@@@@@@@@@@@@@@@@@@@@@@@@@@@@@@@@@@@@@@@@
\chapter{Field Collection and Measurements}
% \addcontentsline{toc}{chapter}{Appendix}
% \renewcommand{\thesection}{\Alph{section}}
% @@@@@@@@@@@@@@@@@@@@@@@@@@@@@@@@@@@@@@@@@@@@@@@@@@@@@@@@@@@@@@@@@@@@@@@@@@@@@
\label{ch:fieldmea}

This part of the appendix describes how to collect the water samples and how to take the different measurements in the field.
% -----------------------------------------------------------------------------
\section{Water Samples}
\todo{add filtration procedure}The water samples should be taken using Dark Nalgene bottles (or similar) simultaneously with in-water optical measurements \citep{Mitchell2002}.

\subsection{Equipment}
\subsubsection*{Acquisition}

\begin{multicols}{3}
\begin{itemize}[itemsep=2pt,parsep=2pt]
  \item Dark Nalgene bottles
  \item Monroe County Environmental Lab bottles
  \item Cooler
  \item Marker
  \item Bottle label
  \item Ice packs
  \item GPS
  \item Extra batteries GPS
  \item Data sheet
  \item Pen
  \item Back pen
  \item Canoe 
  \item Transport straps for canoe
  \item Paddles
  \item Life jacket
  \item Suncream 
  \item Drinking water
  \item Wipes to clean extra suncream from hands
  \item Bucket with rope (in case it is not possible to take water samples due to bad weather conditions, for example)
\end{itemize}
\end{multicols}

\subsubsection*{Filtration}
\begin{itemize}[itemsep=2pt,parsep=2pt]
  \item Whatman Binder-Free Glass Microfiber Filters: Type GF/F - Diameter: 4.7cm
  \item Vacuum pump
  \item Balance
  \item Graduate cylinder
  \item Forceps
\end{itemize}

\subsection{Procedure}
\begin{enumerate}[itemsep=2pt,parsep=2pt]
  \item Throughly clean the bottles prior to collection by brushing them inside and rinsing with tap water a couple of times. Then, allow them to dry
  \item Make sure to take enough bottles for each sample. For low concentration waters, three $1[L]$ bottles is recommended, and for high concentration waters, two $1[L]$ bottles is recommended
  \item Distribute the bottles in the different coolers and fill with enough ice to keep the samples cool
  \item Once in the site, press GPS button to save location. Fill the ``GPS WAYPOINT'' cell in the log sheet with the location number from the GPS
  \item Fill the log sheet in the ``Location Description'', and ``Time''
  \item Take a bottle from the cooler and write the bottle label down in the ``Bottle Number'' section on the data sheet along
  \item Rinse the Nalgene Bottle and cap at least 3 times with the water to be sampled before filling
  \item Submerge bottle with the cap on it in an undisturbed location.
  \item Uncap the submerged bottle  to take subsurface water sample (avoid to take water from the surface) \citep{Montana08} 
  \item Cap the bottle with the bottle still submerged
  \item Store bottle up-right immediately in the cooler in order to avoid direct sun light \citep{Mueller1995}
  \item Once off the water, text to Nina or person in charge of the collection for example: ``Safe, Long Pond team''
  \item Take the in-water optical measurements simultaneously
  \item Take water some water samples to the Monroe County Environmental Lab, if applicable.
  \item Place the sample bottles in the refrigerator as soon as possible
  \item Filter water samples right after collection to preserve the chlorophyll and storage filters in the a $-80[F^\circ]$ freezer as soon as possible
\end{enumerate}
Notes: 
\begin{itemize}[itemsep=2pt,parsep=2pt]
  \item Never try to take samples alone. The team should be of at least two people
  \item Do not take any personal electronic device with you (recommended) to avoid dropping it on the water
  \item Storage car keys in zipped bag in you pocket
  \item In case of bad weather conditions that do not allow paddle the canoes, take at least water samples from the Charlotte pier with the bucket
\end{itemize}

% -----------------------------------------------------------------------------
\section{Remote-sensing Reflectance (\texorpdfstring{$R_{rs}$}{Rrs})}

% method described by \citet{Mobley:1999} for measuring the spectra of the downwelling irradiance $E_d$, the surface reflected sky radiance $L_s$, and the water-leaving radiance $L_w$ for each site 

This section describes how to take the remote-sensing measurement using a single instrument that measures radiance (spectroradiometer or spectrometer) such as a SVC \citep{SVCHR1024i} or an ASD \citep{ASDManual2012} instrument. This procedure is taken from \citet{Mobley:1999} and \citet{Mueller1995}. 

Recall that the \gls{rrs} (\autoref{eq:Rrs}) is defined as

\begin{equation}\label{eq:RrsAppendix}
	R_{rs}(\theta,\phi,\lambda)=\frac{L_w(\theta,\phi,\lambda)}{E_d(\lambda)}
\end{equation}
where $L_w$ is the water-leaving radiance in the polar and azimuthal directions $\theta$ and $\phi$, respectively, and $E_d$ is the downwelling spectral plane irradiance incident onto the water surface. A radiometer pointing down toward the water surface in direction $(\pi-\theta,\phi)$ does not directly measure $L_w$. Instead, it measures $L_w$  plus any incident sky radiance reflected $L_r$ by the water surface into the field of view of the sensor. This total radiance at the sensor $L_t$ is defined as

\begin{equation}\label{eq:Lt}
	L_t(\theta,\phi) = L_r(\theta,\phi)+L_w(\theta,\phi)\Rightarrow L_w(\theta,\phi)=L_t(\theta,\phi) - L_r(\theta,\phi)
\end{equation}

The term $L_r$ can be replaced by

\begin{equation}\label{eq:Lsky}
	L_r = \rho L_{sky}
\end{equation}
where $\rho$ is the proportionality factor that relates the radiance measured when the sensor views the sky to the reflected sky radiance measured when the sensor views the water surface. \citet{Mobley:1999} suggests to use $\rho \approx 0.028$ for a sensor view angle $\theta_v \approx 40^\circ$ from the nadir and  $\phi_v \approx 135^\circ$ from the Sun with the constraints of a clear sky and wind speed less than $5m/s$.

Although $E_d$ could be measured directly with an appropriate sensor, it will be estimated from the radiance measured from a Lambertian surface (Spectralon) because both instruments in this case (SVC and ASD) are set to measure radiance. When an irradiance $E_d$ falls on a Lambertian surface with a known irradiance reflectance $R_g$, the uniform radiance $L_g$ leaving the surface is given by 

\begin{equation}\label{eq:Lg}
	L_g = (R_g/\pi)E_d\Rightarrow E_d = L_g*\pi/R_g
\end{equation}

Applying \autoref{eq:Lt}, \autoref{eq:Lsky} and \autoref{eq:Lg} in \autoref{eq:RrsAppendix} yields

\begin{equation}\label{eq:RrsFinalAppendix}
	R_{rs} = \frac{L_t-\rho L_{sky}}{\frac{\displaystyle \pi}{\displaystyle R_g}L_g}
\end{equation}

Therefore, from \autoref{eq:RrsFinalAppendix}, there are three quantities that need to be measured in a consecutive order: $L_g$, $L_t$, and $L_{sky}$. The procedures for taking these three measurements in the field will be explained for both the ASD and the SVC instrument.

% ------------------------------------
\subsection{\texorpdfstring{$R_{rs}$}{Rrs} measurement using the ASD}

The ASD instrument should be the ``radiance mode''. Three different radiance measurements need to be taken:
\begin{itemize}[itemsep=2pt,parsep=2pt]
	\item $L_g$: pointing at the Spectralon

Description: $L_g$ is measured with the sensor pointing downward in the same direction as is used in viewing the water surface (see \autoref{fig:Rrsinsitu}.(a)), while the Spectralon is inserted into the sensor FOV. The Spectralon should be normal to the water surface.

	\item $L_t$: pointing at the water surface

Description: $L_t$ is measured with the sensor pointing downward toward the water surface in the direction $\approx \pi-\theta_v = 140^\circ$ from nadir with $\theta_v = 40^\circ$ and $\phi_v \approx 135^\circ$ or $\phi_v \approx -135^\circ$ from the Sun as illustrated in \autoref{fig:Rrsinsitu}.(a).

\begin{figure}[htb]
\begin{minipage}[c]{1.0\linewidth}
\centering
    \includegraphics[width=10cm]{/Users/javier/Desktop/Javier/PHD_RIT/LDCM/WaterQualityProtocols/Latex/Images/Lgmea.png}
    \centerline{(a) $L_t$ measurement}\medskip
\end{minipage}  
\hfill 
    % \vspace{0.5cm}
   % \caption[]{\label{fig:Ltmea} $L_t$ measurement.}
% \centering
\begin{minipage}[d]{1.0\linewidth}
\centering
    \includegraphics[width=10cm]{/Users/javier/Desktop/Javier/PHD_RIT/LDCM/WaterQualityProtocols/Latex/Images/Lskymea.png}
    \centerline{(b) $L_{sky}$ measurement}\medskip
\end{minipage}    
    % \vspace{0.5cm}
   \caption[Rrs {\it in situ} measurement]{\label{fig:Rrsinsitu} $R_{rs}$ measurement.}
\end{figure}

	\item $L_{sky}$: pointing at the sky

Description: $L_{sky}$ is measured with the sensor pointing upward toward the sky in the direction $\approx \theta_v = 40^\circ$ from nadir and $\phi_v \approx 135^\circ$ or $\phi_v \approx -135^\circ$ from the Sun as illustrated in \autoref{fig:Rrsinsitu}.(b).



\end{itemize}

% ------------------------------------
\subsection{\texorpdfstring{$R_{rs}$}{Rrs} measurement using the SVC}
The same three radiance measurements described above need to be taken:

\begin{itemize}[itemsep=2pt,parsep=2pt]
	\item $L_g$: pointing at the Spectralon

Description: The measurement is taken in the same fashion described in the previous section and it is taken only once per site. When the SVC instrument is used in ``reflectance mode'', it is necessary to measure first a standard measurement (Spectralon measurement). This standard measurement is the $L_r$ and is recorded internally in the ``sig'' file. Therefore, $L_r$ needs to be extracted from the later from the ``sig'' file. 

	\item $L_t$: pointing at the water surface

Description: $L_t$ is measured in the same fashion described in the previous section. However, this measurement is saved internally in the ``sig'' file after the standard measurement column ($L_g$).

	\item $L_{sky}$: pointing at the sky	

Description: $L_{sky}$ is measured in the same fashion described in the previous section. However, this measurement is saved internally in the ``sig'' file after the standard measurement column ($L_g$).

\end{itemize}
{\bf Notes:}
\begin{itemize}[itemsep=2pt,parsep=2pt]
  \item A good way to find these angles in the field is to find your shadow, and point the instrument $-45^\circ$ or $45^\circ$ from it, which is equivalent to $\phi_v \approx 135^\circ$ or $\phi_v \approx -135^\circ$.

	\item Wear dark clothes, preferable black, to avoid contamination from adjacent objects.

	\item Avoid any reflection from nearby objects in the boat or ship by covering the ship's side with a black tarp. 
\end{itemize}

Examples of {\it in situ} spectral remote-sensing reflectance $R_{rs}$ for the 09-29-2014 scene are shown in \autoref{fig:insituRrs}.

\begin{figure}[htbp!]
      \centering
      \includegraphics[trim=0 0 0 0,clip,width=9.0cm]{/Users/javier/Desktop/Javier/PHD_RIT/Latex/ThesisPhD/Images/RrsExamplesField.png}  
      \caption[{\it In situ} remote-sensing reflectance for the 09-29-2014 scene]{{\it In situ} remote-sensing reflectance for the different sites on the 09-29-2014 scene . \label{fig:insituRrs}}
\end{figure}





% @@@@@@@@@@@@@@@@@@@@@@@@@@@@@@@@@@@@@@@@@@@@@@@@@@@@@@@@@@@@@@@@@@@@@@@@@@@@@
\chapter{Lab Measurements}
\label{ch:labmea} 

This part of the appendix describes how to take the lab measurements from the water samples. \autoref{fig:ProtocolsDiagram} shows a diagram of the different methods. These measurements included concentration and spectral absorption coefficients of particles, dissolved material and phytoplankton (\gls{cpas}).


\begin{figure}[htb]
% \subfloat[]{
\centering
    \includegraphics[width=15cm]{/Users/javier/Desktop/Javier/PHD_RIT/Latex/ThesisPhD/Images/WaterQualityProtocolDiagram_g.png}%}\hspace{0.5cm}
% \subfloat[]{   
%     \includegraphics[width=8cm]{/Users/javier/Desktop/Javier/PHD_RIT/20122_Winter/Instrumentation/report3/Images/SideFluoSpec.jpg}}
    \vspace{0.5cm}
   \caption[]{\label{fig:ProtocolsDiagram} Lab measurement protocols diagram.}
\end{figure}
 %------------- 

% $a_{YS}$ cannot be determined directly. An approximation of $a_{YS}$ may be obtained by a spectrophotometer scan of a filtered sample (\citep{Bukata1995}, p.125). Spectrophotometer used in normal mode do not measure true absorbance but {\color{red} attenuance} because all the scattered light is measured. To overcome this, the cells can be placed close to a wide photomultiplier (\citep{Kirk1983}, p.51).
% -----------------------------------------------------------------------------
\section{IOPs}
%*******************************
The \gls{iops}, specifically the spectral absorption coefficients, are obtained from spectrophotometric measurements of samples prepared from filtration of water samples. These methods are described by \citet{Mitchell2002}. In brief, the total absorption coefficient for natural water can be defined as \citep{Mitchell2002}
\begin{equation}
  a_{total}(\lambda) = a_w(\lambda) + a_p(\lambda) + a_g(\lambda)~~[m^{-1}] 
\end{equation}
\noindent where $a_w(\lambda)$, $a_p(\lambda)$, and $a_g(\lambda)$ are the spectral absorption coefficients of water, particles, and soluble components, respectively. $a_p(\lambda)$ can be decomposed as
\begin{equation}
  a_p(\lambda) = a_\phi(\lambda) + a_d(\lambda)~~[m^{-1}] 
\end{equation}
\noindent where $a_\phi(\lambda)$ and $a_d(\lambda)$ are the spectral absorption coefficients of phytoplankton and de-pigmented particles, respectively. $a_\phi$, $a_d$, $a_g$ are referred to in \autoref{eq:atotal} as $a_{Chl}$, $a_{SM}$ and $a_{CDOM}$, respectively. The main idea is to first measure $a_p(\lambda)$ and $a_d(\lambda)$ from a dual-beam spectrophotometer, and then obtain $a_\phi(\lambda)$ as $a_\phi(\lambda)=a_p(\lambda)$ - $a_d(\lambda)$.

A dual-beam spectrophotometer is an instrument that measures the amount of light that passes a medium, i.e. transmittance, using two paths, one for the reference, and one for the sample. The spectrophotometer used was a Shimadzu \gls{uv}-\gls{vis} recording spectrophotometer, model UV-2100U. One the output of the instrument is the dimensionless quantity \gls{od} (a.k.a. absorbance). 

The water sample is filtered, and a $OD_{fp}(\lambda)$ is measured using the spectrophotometer. $OD_{fp}(\lambda)$ is a measurement of the \gls{od} of the retained particles. Then, the filter is soaked in chemical solvent to extract phytoplankton pigments \citep{Kishino1985cj}. At this point, the filter only retained the de-pigmented particles, and $OD_{fd}(\lambda)$ is measured. The conversion from \gls{od} to absorption coefficients is \citep{Cleveland1993}
\begin{equation}
  a_p = \frac{\displaystyle 2.3 OD_{susp}(\lambda)}{V/A}
\end{equation}
\noindent where 2.3 is a conversion from $\log$ base $10$ to $\log$ base $e$, $V~[m^3]$ is the volume filtered, $A~[m^2]$ is the clearance area of the filter. $OD_{susp}$ is obtained from $OD_{fp}$ or $OD_{fd}$ after a scattering correction for glass-fiber filter proposed by \citet{Cleveland1993} as
\begin{equation}
   OD_{susp}(\lambda) = 0.378 (OD_{fp}(\lambda)-OD_{fp}(750nm)) + 0.523 (OD_{fp}(\lambda)-OD_{fp}(750nm))^2
 \end{equation} 
 \noindent where $OD_{fp}$ is the optical density of the filter before methanol extraction. $a_d$ is obtained in a similar fashion. In the following section, the procedure to measure $OD_{fp}$ and $OD_{fd}$ in the lab is presented.


\subsection{Mineral and Chlorophyll absorption coefficients}

% %*******************************
% \subsection{Minerals absorption coefficients}

%*******************************
\subsubsection{Equipment}
%*******************************
\subsubsection*{Filtration}
\begin{itemize}[itemsep=2pt,parsep=2pt]
  \item Vacuum pump
  \item Filter tower (filter funnel stem, filter base, funnel, filter cup)
  \item Whatman Binder-Free Glass Microfiber Filters: Type GF/F - Diameter: 2.5cm
  \item Forceps
  \item Graduated cylinder
  \item Beaker with purified water or DIW
  \item Dropper or pipette
\end{itemize}
%*******************************
\subsubsection*{Measurement}
\begin{itemize}[itemsep=2pt,parsep=2pt]
  \item Spectrophotometer
  \item Two-lenses support
  \item Squirt bottle with purified water (e.g. \gls{diw})
  \item Purified water
  \item Small pipette 
  \item Hot Methanol
\end{itemize}
%*******************************
\subsubsection*{Lab Safety}
\begin{itemize}[itemsep=2pt,parsep=2pt]
  \item Lab coat
  \item Gloves
  \item Goggles
\end{itemize}
%*******************************
\subsubsection{Procedure}
%*******************************
\begin{enumerate}[itemsep=2pt,parsep=2pt]
  \item Turn the spectrophotometer on at least 30 minutes before measuring
  \item Set the spectrophotometer parameters in the UV-2101PC software menu: Configure $>$ Parameters... to (suggested):
  \begin{itemize}[itemsep=2pt,parsep=2pt]
    \item Measuring Mode: Abs
    \item Wavelength Range [nm]: Start 900 To End 400
    \item Scan Speed: Medium
    \item Slit Width [nm]: 5.0
    \item Sampling interval [nm]: 1.0
    \item Click OK
  \end{itemize}
  \item Select the Serial Port number to be used for communication with the instrument. Go to: Configure $>$ PC Configuration... In the PC Configuration Parameters, select Photometer Serial Port number where the instrument is connected and click OK
  \item Initialize the instrument. From the menu, go to Configure $>$ Utilities. In the System Utilities window, Turn Photometer On and press OK. A new initialization window that check the system pops up. Wait until it is finished. Note: make sure the beam paths are empty before initializing the instrument
  \item Pour DIW water to two GF/F Whatman filters and stick them in the two-lenses support. Both filter should have the same amount of water. Add water with the pipette or the squirt bottle
  \item Place the two-lenses support in the spectrophotometer
  \item Press the Baseline button in the UV-2101PC software
  \item Perform a scan to see the baseline level of the instrument by pressing the ``Start'' button in the software
  \item Press the ``Go To WL'' button of the software and type $850 [nm]$. Press the ``Auto Zero'' button of the software (optional)
  \item Shake water sample bottle a couple of times to mix by turbulence and ensure large particles that settle at the bottom are re-suspended \citep{Mitchell2002}
  \item \label{item:place_filter} Using the forceps, place the filter on the filter base and place the filter cup on the base. 
  \item Pour the desired amount of water in a graduated cylinder
  \item \label{item:filtration} Turn the vacuum pump on and turn the knob $90^\circ$ to allow filtration. Once all the water pass through the filter, turn the know $90^\circ$ back and the turn the vacuum pump off. \textbf{Record volume filtered}
  \item Using the forceps, take the filter used for the baseline (closer to the front of the instrument) from the two-lenses support and storage it for future baselines. Do not remove the reference filter (closer to the back of the instrument) for the whole measurement session
  \item \label{item:place_filter_spec} Using the forceps, take the sample filter from the filtering tower and stick in the two lenses support. Add one or a few water drops to the sample filter if needed 
  \item  Measure absorbance in the spectrophotometer by pressing the ``Start'' button of the software. This will be the $OD_{fp}$ measurement
  \item Once the scan is finished, input a name for the file and press OK
  \item Using the forceps, remove carefully the sample filter from two lenses support avoiding to break it and place in the filter tower as in step \ref{item:place_filter}
  \item Pour enough solvent (hot methanol) to soak the filter in the filter cup, and filter methanol. The solvent should be disposed in a different flask from the only water filtering
  \item Pour enough solvent (hot methanol) to soak the filter in the filter cup.  Wait 10 min. Check level of solvent frequently and add more if needed. Then, filter as in step \ref{item:filtration}
  \item Pour some purified water and filter again\todo{look reference!}
  \item Repeat step \ref{item:place_filter_spec}
  \item Measure absorbance in the spectrophotometer by pressing the ``Start'' button of the software. This will be the $OD_{fd}$ measurement 
  \item Record the area of filtration in the sample filter
  \item To save measurement, go to File $>$ Data Translation $>$ ASCII Export...  Select channels to be saved
\end{enumerate}

Notes:
\begin{itemize}[itemsep=2pt,parsep=2pt]
  \item Avoid light exposure to the filters when filtering water samples to avoid pigment degradation
  \item The instrument only allows to save 10 measurement, or channels, at the time. 
  \item Add sporadically purified water drops to the reference filter to avoid it to dry
  \item Repeat baseline measurement if needed
\end{itemize}  
{\bf Warning:}  Use of hot methanol is risky due to flammability, and volatility. Extra precautions must be taken!!!
% %*******************************
% \subsubsection{Calculations}


%*******************************
\subsection{CDOM absorption coefficients}
The spectral absorption coefficient for CDOM are calculated as \citep{Mitchell2002}
\begin{equation}
  a_g=2.303 \frac{(OD_s(\lambda)-OD_s(\lambda_{NIR}))}{l}~~[m^{-1}]
\end{equation}
\noindent where $OD_s(\lambda)$ is the \gls{od} of the sample measured with spectrophotometer, $\lambda_{NIR}$ is a wavelength in the \gls{nir} where the absorption by dissolved materials is assumed to be zero, and $l$ the cuvette pathlength in meters.

%*******************************
\subsubsection{Equipment}
%*******************************
\subsubsection*{Filtration}
\begin{itemize}[itemsep=2pt,parsep=2pt]
  \item Whatman GD/X 13 and 25mm Disposable Syringe Filters - Nylon $0.2[\mu m]$ Nylon
  \item Syringe
  \item Small glass beakers or small graduated cylinders
\end{itemize}
\subsubsection*{Measurement}
\begin{itemize}[itemsep=2pt,parsep=2pt]
  \item Spectrophotometer (Shimadzu UV2100V - Dual beam spectrophotometer)
  \item 2 clean quartz glass optical cuvettes (a.k.a. cells)
  \item Purified water or DIW
  \item Optics paper
\end{itemize}
%*******************************
\subsubsection{Procedure}
%*******************************
\begin{enumerate}[itemsep=2pt,parsep=2pt]
  \item \textbf{Turn the Spectrophotometer on at least 30 minutes before measuring}. It needs to be warmed up for optimal measurements
  \item Wash the syringe filter out 3 times with purified water
  \item Filter the water samples in advance with the syringe placing the filtered samples in the small glass beakers or small graduated cylinders.
  \item Allow to reach room temperature to avoid bumps in the spectral signal in the NIR (these bumps are produced for temperature difference between the blank and the sample cells)
  \item Rinse cells a couple of times with a small amount of ethanol or acetone by shaking it (optional, if the cells seem dirty)
  \item Use cotton sweep to clean internal face (optional, if face seems dirty)
  \item Rinse cell with purified water
  \item Clean and dry the external surface of the cells with optics paper. Be careful with scratching the surface, specially the front and bottom faces
  \item Select a Slit Width equal to $5.0~[nm]$ in the photometer software
  \item Select a Sampling Interval of $1~[nm]$ or $2~[nm]$ in the photometer software
  \item Fill both cells with purified water and extract bubbles
  \item Place both blank and sample cells filled with purified water in the sample compartment of the spectrophotometer
  \item Press ``Auto Zero'' button in the spectrophotometer software
  \item Press ``Baseline'' button in the spectrophotometer software
  \item Fill sample cell with filtered sample water from the small glass beakers or small graduated cylinders \label{item:fillsample}
  \item Press start button to start scan. This measurement is $OD_s(\lambda)$
  \item Repeat from step \ref{item:fillsample} for all samples
  \item Save Channel in the spectrophotometer software. Go to Data Translation $>$ ASCII Export in spectrophotometer software. The spectrophotometer software can only store a maximum number of ten scans 
\end{enumerate}
\textbf{Important:} the samples should be at room temperature. The \gls{od} measurement is sensible to temperature changes. Filter the water samples in advance and let them reach room temperature. Be aware that it could take hours.

Examples of absorption coefficients measured in the lab are shown in \autoref{fig:IOPlabmea} for the three CPAs.

\begin{figure}[htb!]
\centering
     \includegraphics[trim=0 0 0 0,clip,width=8.0cm]{/Users/javier/Desktop/Javier/PHD_RIT/Latex/ThesisPhD/Images/IOPExamplesField.png} 
\caption{Example of absorption coefficients measured in the lab. \label{fig:IOPlabmea}}
\end{figure}     
% %*******************************
% \subsubsection{Data treatment}
%*******************************

% -----------------------------------------------------------------------------
\section{Concentrations}

% N [69  70]  [71  71]  [71  56]
% RMSE 13.70   8.53    10.92 
% Chl corr    Chl uncorr    TSS
%^^^^^^^^^^^^^^^^^^^  FIGURE ^^^^^^^^^^^^^^^^^^^^^^^^^^^^^^^^^^^^^^^^^^^^
\begin{figure}[htb!]
  \begin{minipage}[c]{0.48\linewidth}
  \centering
  % CHL
      \begin{overpic}[trim=0 0 0 0,clip,width=7cm]{/Users/javier/Desktop/Javier/PHD_RIT/Latex/ThesisPhD/Images/ChlCompRIT_County.pdf}  
  \put (90,70) {(a)}
      \end{overpic}   
  \end{minipage}
  % TSS
  \begin{minipage}[d]{0.48\linewidth}
  \centering
      \begin{overpic}[trim=0 0 0 0,clip,width=7cm]{/Users/javier/Desktop/Javier/PHD_RIT/Latex/ThesisPhD/Images/ChlCompRIT_CountyUncorr.pdf}
  \put (90,70) {(b)}
      \end{overpic}   
  \end{minipage}

  \vspace{0.7cm}
  % CDOM
  \begin{minipage}[c]{1.0\linewidth}
  \centering
      \begin{overpic}[trim=0 0 0 0,clip,width=7cm]{/Users/javier/Desktop/Javier/PHD_RIT/Latex/ThesisPhD/Images/TSSCompRIT_County.pdf}  
  \put (90,70) {(c)}
      \end{overpic}   
  \end{minipage}

  \caption{Comparison of Monroe County Lab versus RIT measurements for (a) chlorophyll-{\it a} corrected for pheophytin, (b) uncorrected chlorophyll-{\it a}, and (c) TSS. \label{fig:RIT_County_Comp} } 
\end{figure}

% ------------------------------
\subsection{Chlorophyll-{\it a} concentration}

There are two kinds of spectrometric methods to determine chlorophyll concentrations. The first methods take into account the presence of pheophytin-{\it a}, which is a common degradation product of chlorophyll-{\it a}, that can interfere with the determination of chlorophyll-{\it a} because it absorbs light and fluoresce in the same region of the spectrum as does chlorophyll-{\it a} \citep{StandardMethodsWater2011}. Examples of these methods are described by \citet{Lorenzen:1967fk}. These methods use a solvent for extraction, $90\%$ acetone in this case. The pheophytin-{\it a} concentration is estimated by adding acid to the chlorophyll-{\it a}, which results in loss of the magnesium atom, converting the chlorophyll-{\it a} to pheophytin-{\it a}. The calculations used by \citep{Lorenzen:1967fk} are:

\begin{equation}
  C_a = \frac{26.7(655_o - 665_a)\times v}{V\times l}
\end{equation}

\begin{equation}
  Pheo = \frac{26.7([1.7\times 665_a]-665_o)\times v}{V\times l}
\end{equation}

\noindent where: \\
$665_o = 665 - (750-blank~value)$ before acidification\\
$665_a = 665 - (750-blank~value)$ after acidification  \\
$v = $ volume of extract in mililiters $[ml]$ \\
$V = $ volume of water filtered in liters $[L]$ \\
$l = $ pathlength of cuvette ($1cm$ for the cuvette used) \\

The second kind of spectrophotometric methods do not take into account the pheophytin-{\it a} presence, and they are based only on the absorbance values at specific wavelengths (a.k.a. uncorrected methods). These are empirical methods that use non linear least squares fitting methods over a dataset with known concentrations. Examples of these methods are described by \citet{Ritchie:2008eu}. One algorithm to determine chlorophyll-{\it a} concentration with $90\%$ acetone as solvent that uses blank-corrected absorbances measured at four wavelengths (quadrichroic) is \citep{Ritchie:2008eu}
\begin{equation}
  C_a = -0.3319A_{630}-1.7485*A_{647}+11.9442A_{664}-1.4306A_{691} (\pm 0.0020)
\end{equation}

\begin{equation}
  Chlorophyll a, mg/m^3 = \frac{C_a\times extract~volume, L}{volume~of~sample, m^3}
\end{equation}
% SM 10200 H, 17th – 20th editions & on-line (2001) edition, published by APHA
% Chl a (10200Hb) and Chl a (10200Hc)
% =$Y9*$E9/$D9
% =-0.3319*$F9-1.7485*$G9+11.9442*$H9-1.4306*$J9

Similar spectrophotometric methods and other kinds of methods are also described in \citet{StandardMethodsWater2011}.
%*******************************
% ------------------------------
\subsubsection[Equipment]{Equipment \citep{Tyler2013_chl_protocol}}
\begin{multicols}{2}
\subsubsection*{Filtration}
\begin{itemize}[itemsep=0.01cm,leftmargin=*]
  \item Vacuum pump
  \item Filter tower
  \item Glass fiber filters: Whatman $4.7cm$, $0.7\mu m$
  \item Graduate cylinder
  \item Forceps
  \item 15 ml Nalgene centrifuge tubes with caps
  \item Drill
  \item Teflon pestle and glass mortar
  \item Cold $90\%$ acetone (spectrophotometric grade)
  \item Squirt bottle with purified water or DIW
  \item Optics paper
  \item Aluminum foil envelopes
  \item MgCO$_3$ slurry (1g/100ml - optional)
\end{itemize}
\subsubsection*{Measurement}
\begin{itemize}[itemsep=0.01cm,leftmargin=*]
  \item Spectrophotometer
  \item 2 cuvettes
  \item Centrifuge
  \item $5\%$ HCl
  \item Pipette

\end{itemize}
\end{multicols}
% ------------------------------
\subsubsection{Procedures}
These procedures are described by \citet{Tyler2013_chl_protocol}.
\subsubsection*{Filtration}
\begin{enumerate}
  \item Using forceps, place glass fiber filter onto tower, rough side up
  \item Shake sample well, and pour into graduated cylinder
  {\bf \item Record volume of water filtered}
  \item Turn on the vacuum pump. Vacuum pressure should always stay at or below 7mm Hg to avoid breaking the cells 
  \item Optional: Add 1 ml of MgCO$_3$ slurry to filter after all of the water has passed through
  \item Turn off vacuum pump
  \item Fold filter in half, and place in aluminum foil envelope (or Preti dish)
  \item Place immediately in freezer (-80$^\circ$ is best)
\end{enumerate}

\subsubsection*{Extraction}
\begin{enumerate}
  \item Using forceps, tear up filter and put into glass mortar
  \item Add 6 ml $90\%$ acetone
  \item Using pestle attached to drill, grind sample until it is a slurry
  \item Pour sample into Nalgene test tube
  \item Using 3 additional ml of acetone, rinse mortar and add rinse to test tube
  \item Cap tube, label with volume filtered, acetone volume, and date, and place immediately in freezer over night
\end{enumerate}

\subsubsection*{Spectrophotometric Measurement}
\begin{enumerate}
  \item Turn on spec and let it warm up for 15-20 minutes, set the wavelength to 665 nm
  \item Pipette 3 ml of $90\%$ acetone into each cuvette, place in the spec
  \item Rezero the instrument
  \item Change the wavelength to 750 nm and record the blank value
  \item Keep the blank cuvette on the instrument and empty the sample cuvette into a waste bottle, rinse once with DIW, and once with $90\%$ acetone
  \item Remove samples tubes from freezer and shake to resuspend
  \item Centrifuge samples tubes for 5 minutes
  \item Shake samples tubes to remove filter particles from the wall of the tubes
  \item Centrifuge samples tubes again for 10 minutes
  \item Pipette 3 ml of sample into the sample cuvette
  \item Read absorbance at 630, 647, 664, 665, 691, and 750 nm
  \item Add two drops of $5\%$ HCl to the cuvette and wait 1 minute
  \item Read absorbance at 665 and 750 nm
\end{enumerate}


\subsection{Total suspended solids (TSS)}

\subsubsection{Equipment}
%*******************************
% \subsubsection*{Filtration}
\begin{itemize}[itemsep=2pt,parsep=2pt]

  \item TCLP filters ($47 mm$, $0.7\mu m$) or membrane of glass fiber filter -- GF/F is standard
  \item Vacuum pump
  \item Filter apparatus
  \item Petri dishes or aluminum foil envelopes
  \item Squirt bottle with DIW
  \item balance
  \item Graduate cylinder
  \item Forceps

\end{itemize}

%*******************************
\subsubsection{Procedure}
%*******************************
\begin{enumerate}[itemsep=2pt,parsep=2pt]
  \item Weigh filters before filtering
  \item Store weighed filter in aluminum foil with label including weight
  \item Shake sample well, and measure sample in a graduated cylinder. 
  \item {\bf Record volume filtered}
  \item Using forceps, place a pre-weighed filter (if GF/F, rough side up) on the filter tower
  \item Pour sample water from the graduated cylinder into filter tower and turn vacuum pump
  \item After sample has filtered, place in aluminum foil using the forceps and label including sample name and date
  \item If not analyzing samples right away, place in freezer for storage
  \item Dry sample at $60^\circ C$ for a couple of hours or overnight. Note: Check to see if the samples are dry by checking the weight at two different times
  \item Weigh dried filter in the balance
\end{enumerate}
{\bf Notes:}
\begin{itemize}[itemsep=2pt,parsep=2pt]
  \item If the organic fraction (SPOM) of the TSS is desired, the filters need to be combusted in a muffle furnace at $450^\circ C$ for 4 before being weighed again
  \item The weight should be obtained using a 6 place balance and recorded on a data sheet
\end{itemize}


 %*******************************
\subsubsection{Calculations}
From \citet{Tyler2013_SPM_protocol}:

\begin{equation}
TSS~or~SPM = \frac{[final~filter~weight~(mg) - tare~filter~weight~(mg)]}{volume~filtered~(L)}
\end{equation}

\begin{equation}
SPOM = \frac{[final~filter~weight~(mg) - AFDM~(mg)]}{volume~filtered~(L)}
\end{equation}

\noindent Note: concentrations are in $[mg/m^3]$ or $[\mu g/L]$.

% &&&&&&&&&&&&&&&&&&

% @@@@@@@@@@@@@@@@@@@@@@@@@@@@@@@@@@@@@@@@@@@@@@@@@@@@@@@@@@@@@@@@@@@@@@@@@@@@@
\chapter{Main Codes}

The codes used in this research can be found at \url{https://github.com/javierconcha}.

\singlespacing
\lstset{language=bash,rulecolor=\color{black},caption={Example of an input file used in Ecolight.},label=code:EcolightInput}
\renewcommand{\lstlistingname}{Code}
\begin{lstlisting}
0,400,2500,.02,488,.00026,1,5.3
FFbb determination for ONTOS
OutputEL
0,1,0,0,0,1
2,1,0,2,3
4,4
0,flaCH,flaCD,flaSM
0,2,440,0.1,0.014
0,0,440,0.1,0.014
0,4,440,1,0.01712
0,0,440,0.1,0.014
/home/jxc4005/hydrolight52Javier_install/data/H2OabDefaults_FRESHwater.txt
/home/jxc4005/HYDROLIGHT/EL5.2/user_inputs/astar_CH_ONTOS140929_CountyUncorr.txt
dummyastar.txt
/home/jxc4005/HYDROLIGHT/EL5.2/user_inputs/astar_SM_ONTOS140929_County.txt
4, 660, 0.189, 1, 0.751, -999
0,-999,-999,-999,-999,-999
-1,-999,0,-999,-999,-999
0,-999,-999,-999,-999,-999
bstarDummy.txt
/home/jxc4005/HYDROLIGHT/EL5.2/user_inputs/ChloroSct.txt
dummybstar.txt
/home/jxc4005/HYDROLIGHT/EL5.2/user_inputs/susmin.sct
0, 0, 550, 0.01, 0
0, 0, 0, 0, 0
-1, 0, 0, 0, 0
0, 0, 550, 0.01, 0
pureh2o.dpf
user_dpfCHL
isotrop.dpf
user_dpfTSS_b
 120
400, 405, 410, 415, 420, 425, 430, 435, 440, 445,
450, 455, 460, 465, 470, 475, 480, 485, 490, 495,
500, 505, 510, 515, 520, 525, 530, 535, 540, 545,
550, 555, 560, 565, 570, 575, 580, 585, 590, 595,
600, 605, 610, 615, 620, 625, 630, 635, 640, 645,
650, 655, 660, 665, 670, 675, 680, 685, 690, 695,
700, 705, 710, 715, 720, 725, 730, 735, 740, 745,
750, 755, 760, 765, 770, 775, 780, 785, 790, 795,
800, 805, 810, 815, 820, 825, 830, 835, 840, 845,
850, 855, 860, 865, 870, 875, 880, 885, 890, 895,
900, 905, 910, 915, 920, 925, 930, 935, 940, 945,
950, 955, 960, 965, 970, 975, 980, 985, 990, 995,
1000,
0,0,0,0,2
2, 3, 48, 0, 0
272, 43.28085,-77.61919, 29.92, 1, 80, 2.5, 15, 4.99746, 300
4.99746, 1.34, 20, 35
0, 0
0, 5, 0, 5, 10, 15, 20, 
/home/jxc4005/hydrolight52Javier_install/data/H2OabDefaults_FRESHwater.txt
1
/home/jxc4005/hydrolight5Aaron_install/data/user/mascot_ac9.txt
dummyFilteredAc9.txt
dummyHscat.txt
/home/jxc4005/hydrolight5Aaron_install/data/user/Chlzdata_10m.txt
dummyComp.txt
dummyR.bot
dummydata.txt
/home/jxc4005/hydrolight5Aaron_install/data/user/Chlzdata_10m.txt
dummyComp.txt
dummyComp.txt
/home/jxc4005/hydrolight5Aaron_install/data/user/Ed_total.txt
/home/jxc4005/hydrolight5Aaron_install/data/MyBiolumData.txt
\end{lstlisting}
% @@@@@@@@@@@@@@@@@@@@@@@@@@@@@@@@@@@@@@@@@@@@@@@@@@@@@@@@@@@@@@@@@@@@@@@@@@@@@
\singlespacing
\lstset{language=matlab,rulecolor=\color{black},caption={Optimization Routine in MATLAB.},label=code:optimization routine}
\renewcommand{\lstlistingname}{Code}
\begin{lstlisting}
% Optimization Routine
function [XResults,residual,IMatrix] = opt(Ytest,LUT,LUTconc,LUTconcInput,LUTconcDPF)
format short;

% Xtest: water pixels concentration from the image; Dim: 2000x3
% Ytest: water pixels reflectance from the image; Dim: 2000x8
% LUT: LUT

% global visual
% global visual2
%% Reading in the LUT
Y = LUT;

%% Optimization
% [f2,f1,f3] = ndgrid(CDOMconc,SMconc,CHLconc); % CDOM SM CHL
options = optimset('Display','off','Tolfun',1e-10);

tic

matlabpool open 4 % for using paralel computing

XResults    = zeros(size(Ytest,1),3);
residual    = zeros(size(Ytest));
IMatrix     = zeros(size(Ytest,1),1);
%%

parfor i = 1:size(Ytest,1)
%     if visual==1
% figure(30)
% clf
% 
% i
% if i==172
%    disp('i=172') 
% end

%     figure(68)
%     clf
%     ylim([0 0.05])
%     end 
%     
%     if visual2==1
%     figure(69)
%     xlim([0 68])
%     ylim([0 24])
%     zlim([0 14])
%     end
    

    % select x0
    a=sum((Y-repmat(Ytest(i,:),size(Y,1),1)).^2,2);
    [~,index]=min(a);
    x0=LUTconc(index,:);

    % Select from the LUT with same input. From ponds OR lake inputs
    cond1 = LUTconcInput == LUTconcInput(index); %fix pond or lake
    cond2 = LUTconcDPF == LUTconcDPF(index); % fix phase function
    cond3 =  cond1&cond2;
    
    IMatrix(i) = index; % index in the LUT
    
    CDconc  = unique(LUTconc(cond3,3))';
    SMconc  = unique(LUTconc(cond3,2))';
    CHconc  = unique(LUTconc(cond3,1))';
    
    LUTconcUsed = LUTconc(cond3,:);
    YUsed = Y(cond3,:);
    
    % for the extremes (myfun_mod.m error otherwise)
    if x0(1)==max(CHconc) , x0(1)=CHconc(end-1); end
    if x0(2)==max(SMconc) , x0(2)=SMconc(end-1); end
    if x0(3)==max(CDconc) , x0(3)=CDconc(end-1); end
    
    % Y: LUT
    % Ytest: TestSamples
    % f1,f2,f3: Grid for the LUT
    [XResults(i,:),~,residual(i,:)]= ...
        lsqnonlin(@MyTrilinearInterp,x0,...
            [min(CHconc);min(SMconc);min(CDconc)],...
            [max(CHconc);max(SMconc);max(CDconc)],...
            options,YUsed,Ytest(i,:),LUTconcUsed);
    
end


matlabpool close % for using paralel computing

disp('Elapsed time is (min):')
disp(toc/60)
beep
pause(0.1)
beep
pause(0.1)
beep
pause(0.1)
beep
\end{lstlisting}




% @@@@@@@@@@@@@@@@@@@@@@@@@@@@@@@@@@@@@@@@@@@@@@@@@@@@@@@@@@@@@@@@@@@@@@@@@@@@@
\singlespacing
\lstset{language=matlab,rulecolor=\color{black},caption={"MyTrilinearInterp.m": Trylinear interpolation for the optimization routine.},label=code:trylinearinterp}
\renewcommand{\lstlistingname}{Code}
\begin{lstlisting}
function f = MyTrilinearInterp(x0,LUT,Ytest,LUTconc)
% Trilinear interpolation
% By Javier A. Concha
% 05-09-13
% Trilinear interpolation for non uniform and monotoning grid
%% Intialization
% Initial Concentration

nx = x0(1); 
ny = x0(2);
nz = x0(3);

% Concentrations per components
xx = unique(LUTconc(:,1)); % CHL
yy = unique(LUTconc(:,2)); % SM
zz = unique(LUTconc(:,3)); % CDOM
%% To find index in between xx and weight for each dim
index_up = find(xx >= nx,1);

if isempty(index_up)
    index_up = length(xx);
end

index_low = index_up-1;

if index_low ~= 0
    xl = xx(index_low); % lower concentration
else xl = 0;
end
xu = xx(index_up);      % upper concentration

wx = (nx-xl)/(xu-xl);
if isnan(wx)
    wx = 1;
end
%% To find index in between yy and weight for each dim
indey_up = find(yy >= ny,1);

if isempty(indey_up)
    indey_up = length(yy);
end

indey_low = indey_up-1;

if indey_low ~= 0
    yl = yy(indey_low);
else yl = 0;
end
yu = yy(indey_up);


wy = (ny-yl)/(yu-yl);
if isnan(wy)
    wy = 1;
end
%% To find index in between zz and weight for each dim
indez_up = find(zz >= nz,1);

if isempty(indez_up)
    indez_up = length(zz);
end

indez_low = indez_up-1;

if indez_low ~= 0
    zl = zz(indez_low);
else zl = 0;
end
zu = zz(indez_up);

wz = (nz-zl)/(zu-zl);
if isnan(wz)
    wz = 1;
end
%% Look up the values of the 8 points surrounding the cube
if (index_low ~= 0 && indey_low ~= 0 && indez_low ~= 0)
    V000 = LUT(LUTconc(:,1)==xl & ...
        LUTconc(:,2)==yl & ...
        LUTconc(:,3)==zl,:);
else
    V000 = zeros(1,size(LUT,2));
end

if (index_low ~= 0 && indey_low ~= 0)
    V001 = LUT(LUTconc(:,1)==xl & ...
        LUTconc(:,2)==yl & ...
        LUTconc(:,3)==zu,:);
else
    V001 = zeros(1,size(LUT,2));
end

if (index_low ~= 0 && indez_low ~= 0)
    V010 = LUT(LUTconc(:,1)==xl & ...
        LUTconc(:,2)==yu & ...
        LUTconc(:,3)==zl,:);
else
    V010 = zeros(1,size(LUT,2));
end

if index_low ~= 0
    V011 = LUT(LUTconc(:,1)==xl & ...
        LUTconc(:,2)==yu & ...
        LUTconc(:,3)==zu,:);
else
    V011 = zeros(1,size(LUT,2));
end

if (indey_low ~= 0 && indez_low ~= 0)
    V100 = LUT(LUTconc(:,1)==xu & ...
        LUTconc(:,2)==yl & ...
        LUTconc(:,3)==zl,:);
else
    V100 = zeros(1,size(LUT,2));
end

if indey_low ~= 0
    V101 = LUT(LUTconc(:,1)==xu & ...
        LUTconc(:,2)==yl & ...
        LUTconc(:,3)==zu,:);
else
    V101 = zeros(1,size(LUT,2));
end

if indez_low ~= 0
    V110 = LUT(LUTconc(:,1)==xu & ...
        LUTconc(:,2)==yu & ...
        LUTconc(:,3)==zl,:);
else
    V110 = zeros(1,size(LUT,2));
end

V111 = LUT(LUTconc(:,1)==xu & ...
    LUTconc(:,2)==yu & ...
    LUTconc(:,3)==zu,:);

Vxyz =  ...
    V000*(1-wx)*(1-wy)*(1-wz) +...
    V001*(1-wx)*(1-wy)*wz + ...
    V010*(1-wx)*wy*(1-wz) + ...
    V011*(1-wx)*wy*wz + ...
    V100*wx*(1-wy)*(1-wz) + ...
    V101*wx*(1-wy)*wz + ...
    V110*wx*wy*(1-wz) + ...
    V111*wx*wy*wz;

f = Ytest - Vxyz;


% figure(30)
% plot(Ytest,'r')
% hold on
% plot(Vxyz,'g')
% plot(V000)
% plot(V001)
% plot(V010)
% plot(V011)
% plot(V100)
% plot(V101)
% plot(V110)
% plot(V111)

\end{lstlisting}
% @@@@@@@@@@@@@@@@@@@@@@@@@@@@@@@@@@@@@@@@@@@@@@@@@@@@@@@@@@@@@@@@@@@@@@@@@@@@@

\singlespacing
\lstset{language=bash,rulecolor=\color{black},caption={Bash script to create LUT in Hydrolight.},label=code:LUT creation}
\renewcommand{\lstlistingname}{Code}
\begin{lstlisting}
#!/bin/bash
rm directory_list.txt
rm -r SITE/
rm concentration_list.txt
for SITE in `cat SITE_list.txt`
do
for CHL in `cat CHL_list.txt`
do
      for SM in `cat SM_list.txt`
  do
    for CDOM in `cat CDOM_list.txt`
    do
      for DPF in `cat DPF_list.txt`
      do
      directory="/home/jxc4005/HYDROLIGHT/EL5.2/SITE/"$SITE"/CHL/"$CHL"/SM/"$SM"/CDOM/"$CDOM"/DPF/"$DPF 
      mkdir -p $directory
      echo $directory
      echo $directory >> directory_list.txt
      echo $SITE $CHL $SM $CDOM $DPF>> concentration_list.txt 
      mkdir -p $directory/ref 
      
      let i++
      input=$SITE".txt"   
      cat $input | sed 's"site"'$SITE'"g' | sed 's"flaCH"'$CHL'"g' | sed 's"flaCD"'$CDOM'"g' | sed 's"flaSM"'$SM'"g' | sed 's"user_dpfCHL"'$DPF'"g' | sed 's"user_dpfTSS"'$DPF'"g'  > $directory/input.txt

        cat ELRun.sh | sed 's"flag"'$directory'"g' > $directory/ELRun.sh
      #cp ./CommonInputs/* $directory/
      #sbatch --qos=cis-normal $directory/ELRun.sh
      sbatch --qos free --partition work --mem=12 $directory/ELRun.sh
      #sbatch --qos schott --partition premium $directory/ELRun.sh
    # sbatch --qos schott --partition premium --mem=12 $directory/ELRun.sh
      echo $i
      done
    done
  done
done
done 

\end{lstlisting}

% @@@@@@@@@@@@@@@@@@@@@@@@@@@@@@@@@@@@@@@@@@@@@@@@@@@@@@@@@@@@@@@@@@@@@@@@@@@@@

\singlespacing
\lstset{language=bash,rulecolor=\color{black},caption={"ELRun.sh": Script to submit jobs to SLURM.},label=code:ELRun_sh}
\renewcommand{\lstlistingname}{Code}
\begin{lstlisting}
#!/bin/bash -l
# NOTE the -l flag!
#

# This is an example job file for a multi-core MPI job.
# Note that all of the following statements below that begin
# with #SBATCH are actually commands to the SLURM scheduler.
# Please copy this file to your home directory and modify it
# to suit your needs.
# 
# If you need any help, please email rc-help@rit.edu
#

# Name of the job - You'll probably want to customize this.
#SBATCH -J ELRun

# Standard out and Standard Error output files
#SBATCH -o ELRun.output
#SBATCH -e ELRun.output

# Request 2 mins run time MAX, anything over will be KILLED
#SBATCH -t 0:5:0

# Put the job in the "work" partition and request FOUR cores
# "work" is the default partition so it can be omitted without issue.
#SBATCH -p work -n 1


#
# Your job script goes below this line.  
# 

#/home/jxc4005/ecolight52_install/Code/Ecolight/mainEL_stndg95.exe < ./input.txt

cd flag

/home/jxc4005/ecolight52_install/Code/Ecolight/mainEL_stnd.exe < ./input.txt

awk 'c&&!--c;/Rrs \= Lw\/Ed/{c=3}' POutputEL.txt |awk '{ print $8 }'>tempR.txt

mv tempR.txt ./ref/
#rm ./* 2> /dev/null
#exit

\end{lstlisting}

% @@@@@@@@@@@@@@@@@@@@@@@@@@@@@@@@@@@@@@@@@@@@@@@@@@@@@@@@@@@@@@@@@@@@@@@@@@@@@

\singlespacing
\lstset{language=bash,rulecolor=\color{black},caption={Bash script to extract Rrs from Hydrolight runs.},label=code:Rrs extraction from Hydrolight runs}
\renewcommand{\lstlistingname}{Code}
\begin{lstlisting}
#!/bin/bash
rm ./work/Rvector.txt
#rm file_list.txt
rm concentration_listright.txt
rm wrong_list.txt
rm right_list.txt
#find SITE/ -name "tempR.txt" > file_list.txt
j=0

for DIR in `cat directory_list.txt`
do
  echo $DIR
        SITE="$(echo $DIR | cut -d/ -f7)"
        CHL="$(echo $DIR | cut -d/ -f9)"
        SM="$(echo $DIR | cut -d/ -f11)"
        CDOM="$(echo $DIR | cut -d/ -f13)"
        DPF="$(echo $DIR | cut -d/ -f15)"
        echo $SITE $CHL $SM $CDOM $DPF

  directory="/home/jxc4005/HYDROLIGHT/EL5.2/SITE/"$SITE"/CHL/"$CHL"/SM/"$SM"/CDOM/"$CDOM"/DPF/"$DPF

  if [ " $(cat $DIR/ref/tempR.txt | wc -l)" -ne 120 ]
        then
                echo $DIR >> $PWD/wrong_list.txt
    echo Warning: $DIR not completed yet and resubmitted!
    
    input=$SITE".txt"
                cat $input | sed 's"site"'$SITE'"g' | sed 's"flaCH"'$CHL'"g' | sed 's"flaCD"'$CDOM'"g' | sed 's"flaSM"'$SM'"g' | sed 's"user_dpfCHL"'$DPF'"g' | sed 's"user_dpfTSS"'$DPF'"g'  > $directory/input.txt

                 cat ELRun.sh | sed 's"flag"'$directory'"g' > $directory/ELRun.sh
                 #sbatch --qos=cis-normal $directory/ELRun.sh
     #sbatch --qos=cis-nopreempt $directory/ELRun.sh
                 #sbatch --qos schott --partition premium $directory/ELRun.sh
           sbatch --qos free  --partition work --mem=12 $directory/ELRun.sh
     let j++
  else

                echo $DIR >> $PWD/right_list.txt
                cat $DIR/ref/tempR.txt >> $PWD/work/Rvector.txt
                echo $SITE $CHL $SM $CDOM $DPF>> $PWD/concentration_listright.txt
                #rm $directory/* 2> /dev/null
  fi
        let i++
        echo $i

done
echo Jobs completed correctly: $(cat ./right_list.txt|wc -l)
echo Jobs not completed and resubmitted: $j

if [ "$j" -ne 0 ]
then
  nohup nice ./monitor.sh 2> nohup.out &
else
  exit
fi


\end{lstlisting}


% @@@@@@@@@@@@@@@@@@@@@@@@@@@@@@@@@@@@@@@@@@@@@@@@@@@@@@@@@@@@@@@@@@@@@@@@@@@@@

% \singlespacing
% \lstset{language=matlab,caption={Example of an input file used in Ecolight.},label=code:optimization routine}
% \renewcommand{\lstlistingname}{Code}
% \begin{lstlisting}

% \end{lstlisting}


\end{appendices}